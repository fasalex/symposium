%
\documentclass[conference]{IEEEtran}
\hyphenation{op-tical net-works semi-conduc-tor}


\begin{document}
%
% paper title
\title{Bare Demo of IEEEtran.cls for Conferences}

% author names and affiliations
% use a multiple column layout for up to three different
% affiliations
\author{\IEEEauthorblockN{Fasika Assegei}
\IEEEauthorblockA{Department of Electrical Engineering\\
Eindhoven University of Technology\\
Eindhoven, The Netherlands\\
Email: f.a.assegei@student.tue.nl}}

\maketitle


\begin{abstract}
%\boldmath
The abstract goes here.
\end{abstract}

\section{Introduction}
% no \IEEEPARstart
This demo file is intended to serve as a ``starter file''
for IEEE conference papers produced under \LaTeX\ using
IEEEtran.cls version 1.7 and later.
% You must have at least 2 lines in the paragraph with the drop letter
% (should never be an issue)
I wish you the best of success.

\hfill mds

\hfill January 11, 2007

\subsection{Subsection Heading Here}
Subsection text here.


\subsubsection{Subsubsection Heading Here}
Subsubsection text here.

\section{Conclusion}
The conclusion goes here.




% conference papers do not normally have an appendix


% use section* for acknowledgement
\section*{Acknowledgment}


The authors would like to thank...


\begin{thebibliography}{1}

\bibitem{IEEEhowto:kopka}
H.~Kopka and P.~W. Daly, \emph{A Guide to \LaTeX}, 3rd~ed.\hskip 1em plus
  0.5em minus 0.4em\relax Harlow, England: Addison-Wesley, 1999.

\end{thebibliography}


% that's all folks
\end{document}
